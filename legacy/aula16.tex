% Created 2020-05-05 ter 23:21
% Intended LaTeX compiler: pdflatex
\documentclass[a4paper]{article}

\usepackage{booktabs}
\usepackage[margin=2cm]{geometry}
\usepackage{amsmath,amsfonts,amssymb,amsthm}
\usepackage{sourcecodepro}
\usepackage[utf8]{inputenc}
\usepackage{booktabs}
\usepackage{array}
\usepackage{colortbl}
\usepackage{listings}
\usepackage{algpseudocode}
\usepackage{algorithm}
\usepackage{graphicx}
\usepackage[english, ]{babel}
\usepackage[scale=2]{ccicons}
\usepackage{hyperref}
\usepackage{relsize}
\usepackage{amsmath}
\usepackage{bm}
\usepackage{amsfonts}
\usepackage{wasysym}
\usepackage{float}
\usepackage{ragged2e}
\usepackage{textcomp}
\usepackage{pgfplots}
\usepackage{todonotes}
\usepgfplotslibrary{dateplot}
\lstdefinelanguage{ein-julia}%
{morekeywords={abstract,struct,break,case,catch,const,continue,do,else,elseif,%
end,export,false,for,function,immutable,mutable,using,import,importall,if,in,%
macro,module,quote,return,switch,true,try,catch,type,typealias,%
while,<:,+,-,::,/},%
sensitive=true,%
alsoother={$},%
morecomment=[l]\#,%
morecomment=[n]{\#=}{=\#},%
morestring=[s]{"}{"},%
morestring=[m]{'}{'},%
}[keywords,comments,strings]%
\lstset{ %
backgroundcolor={},
basicstyle=\ttfamily\scriptsize,
breakatwhitespace=true,
breaklines=true,
captionpos=n,
extendedchars=true,
frame=n,
language=R,
rulecolor=\color{black},
showspaces=false,
showstringspaces=false,
showtabs=false,
stepnumber=2,
stringstyle=\color{gray},
tabsize=2,
}
\renewcommand*{\UrlFont}{\ttfamily\smaller\relax}
\author{Alfredo Goldman}
\date{\today}
\title{Alfredo's MAC0110 Journal}
\hypersetup{
 pdfauthor={Alfredo Goldman},
 pdftitle={Alfredo's MAC0110 Journal},
 pdfkeywords={},
 pdfsubject={},
 pdfcreator={Emacs 26.3 (Org mode 9.3.6)},
 pdflang={Bt-Br}}
\begin{document}

\maketitle

\section{Programa do curso}

\subsection{Aula 16 \textit{<2020-05-06 qua>}}
\label{sec:org881c781}
\subsubsection{Um pouco mais de sintaxe de Julia e vetores}
\label{sec:org06f3ee9}

Como é de se imaginar, testes automatizados são muito
utilizados por bons desenvolvedores. Logo podemos imaginar
que a linguagem Julia tem formas de ajudar a escrita de
testes.

Isso é feito com o pacote de testes de Julia, para usá-lo precisamos
usar o comando using. Vamos a um exemplo abaixo:
\lstset{language=ein-julia,label= ,caption= ,captionpos=b,numbers=none}
\begin{lstlisting}
using Test
@test 1 == 1
@test 1 != 2
\end{lstlisting}

 O comando @test avalia a expressão e verifica se o valor é
verdadeiro (true), se for, não faz nada. Mas, se não for, aponta
o erro.

 Dessa forma, podemos escrever testes em Julia de forma mais compacta.
Para começar vamos verificar se uma função que recebe um vetor e devolve
a soma dos seus elementos funciona.

\lstset{language=ein-julia,label= ,caption= ,captionpos=b,numbers=none}
\begin{lstlisting}
using Test
include("soma.jl")
function testaTudo()
  @test soma([]) == 0
  @test soma([1]) == 1
  @test soma([10, 20, 30]) == 60
  println("final dos testes")
end
testaTudo()
\end{lstlisting}
Que funciona verifica se a seguinte função funciona:

\lstset{language=ein-julia,label= ,caption= ,captionpos=b,numbers=none}
\begin{lstlisting}
function soma(v)
  s = 0
  for el in v
    s = s + el
  end
  return s
end
\end{lstlisting}

 Outro comando útil de Julia é a verificação aproximada, pois já
vimos que operações com número reais nem sempre é exata. Essa
comparação é dada com \(\approx\) (barra approx)

\lstset{language=ein-julia,label= ,caption= ,captionpos=b,numbers=none}
\begin{lstlisting}
0.2 + 0.2 + 0.2 \approx 0.6
\end{lstlisting}

\subsubsection{Voltando a vetores}
\label{sec:org4802cdc}

Vamos agora voltar à parte algorítmica, com o seguinte problema.
Subsequência de soma máxima. Dado um vetor de inteiros, devolver a
soma de elementos consecutivos que seja máxima.

Vamos começar pelos testes.

\lstset{language=ein-julia,label= ,caption= ,captionpos=b,numbers=none}
\begin{lstlisting}
using Test
function verificaSoma()
  @test somasub([]) == 0
  @test somasub([1, 2, 3]) == 6
  @test somasub[-1, -2, -3]) == -1
  @test somasub([10, 5, -17, 20, 5, -1, 3, -30, 10]) == 72
  @test somasub([31, -41, 59, 26, -53, 58, 97, -93, -23, 84] == 187
  println("Final dos testes")
end
\end{lstlisting}

 Vamos começar com a solução de força bruta, isso é, calcular a soma
de todas a sub-sequências, procurando pela máxima.

\lstset{language=ein-julia,label= ,caption= ,captionpos=b,numbers=none}
\begin{lstlisting}
# lugar para escrever o codigo
\end{lstlisting}

Agora vamos a um algoritmo mais elaborado. (Jay-Kadane)
\lstset{language=ein-julia,label= ,caption= ,captionpos=b,numbers=none}
\begin{lstlisting}
function somasub(v)
  if length(v) == 0
    return 0
  end
  soma = 0
  somamax = v[1]
  for i in 1:length(v)
    if soma + v[i] < 0
      soma = 0
    else
      soma = soma + v[i]
    end
    if soma > somamax
      somamax = soma
    end
  end
  return somamax
end
\end{lstlisting}

Faça uma função, onde dados dois vetores u e v, devolve o seu
produto escalar.

 Faça uma função, onde dados dois vetores ordenados u e v,
sem repetição, devolve o vetor ordenado com os elementos de u
e v, sem repetição.
\end{document}

% Created 2020-05-17 dom 20:08
% Intended LaTeX compiler: pdflatex
\documentclass[a4paper]{article}

\usepackage{booktabs}
\usepackage[margin=2cm]{geometry}
\usepackage{amsmath,amsfonts,amssymb,amsthm}
\usepackage{sourcecodepro}
\usepackage[utf8]{inputenc}
\usepackage{booktabs}
\usepackage{array}
\usepackage{colortbl}
\usepackage{listings}
\usepackage{algpseudocode}
\usepackage{algorithm}
\usepackage{graphicx}
\usepackage[english, ]{babel}
\usepackage[scale=2]{ccicons}
\usepackage{hyperref}
\usepackage{relsize}
\usepackage{amsmath}
\usepackage{bm}
\usepackage{amsfonts}
\usepackage{wasysym}
\usepackage{float}
\usepackage{ragged2e}
\usepackage{textcomp}
\usepackage{pgfplots}
\usepackage{todonotes}
\usepgfplotslibrary{dateplot}
\lstdefinelanguage{ein-julia}%
{morekeywords={abstract,struct,break,case,catch,const,continue,do,else,elseif,%
end,export,false,for,function,immutable,mutable,using,import,importall,if,in,%
macro,module,quote,return,switch,true,try,catch,type,typealias,%
while,<:,+,-,::,/},%
sensitive=true,%
alsoother={$},%
morecomment=[l]\#,%
morecomment=[n]{\#=}{=\#},%
morestring=[s]{"}{"},%
morestring=[m]{'}{'},%
}[keywords,comments,strings]%
\lstset{ %
backgroundcolor={},
basicstyle=\ttfamily\scriptsize,
breakatwhitespace=true,
breaklines=true,
captionpos=n,
extendedchars=true,
frame=n,
language=R,
rulecolor=\color{black},
showspaces=false,
showstringspaces=false,
showtabs=false,
stepnumber=2,
stringstyle=\color{gray},
tabsize=2,
}
\renewcommand*{\UrlFont}{\ttfamily\smaller\relax}
\author{Alfredo Goldman}
\date{\today}
\title{Alfredo's MAC0110 Journal}
\hypersetup{
 pdfauthor={Alfredo Goldman},
 pdftitle={Alfredo's MAC0110 Journal},
 pdfkeywords={},
 pdfsubject={},
 pdfcreator={Emacs 26.3 (Org mode 9.3.6)},
 pdflang={Bt-Br}}
\begin{document}

\maketitle


\section{Aula 18 - Simulado de prova \textit{<2020-05-20 qua>}}
\label{sec:org648dad0}
\subsection{Prova de 2019 (traduzida de C para Julia)}
\label{sec:org5f1c2ef}
\subsubsection{Questão 1 (1.5 pontos)}
\label{sec:org4d6b3f5}
Dado o seu NUSP qual é a saída do programa abaixo?
\lstset{language=ein-julia,label= ,caption= ,captionpos=b,numbers=none}
\begin{lstlisting}
function misterio(n)
   b = n
   c = -1
   while b > 0
      a = b \% 10
      b = div(b, 10)
      if a > c
         c = a
      end
      x = float(b / 10)
      println("n = ", n, "  a = ", a, "  b = ", b, "  c = ", c, "  x = ", x)
   end
   println("c = ", c, " n/100 ", n/100)
end
\end{lstlisting}

\subsubsection{Questão 2 (2.5 pontos)}
\label{sec:org9dcd571}
Um número inteiro $n > 0$ é perfeito se ele for igual à soma de seus divisores
positivos diferentes de n.

Exemplo:
\begin{itemize}
\item 6 é perfeito, pois 6 = 1 + 2 + 3;
\item 28 é perfeito, pois 28 = 1 + 2 + 4 + 7 + 14.
\end{itemize}

Faça uma função que recebe um número inteiro n > 0 e decide se n é perfeito.

\subsubsection{Questão 3 (2.5 pontos)}
\label{sec:org0996387}
Dado um vetor n números inteiros, desejamos encontrar o comprimento
de um maior segmento crescente da sequência.
Exemplo:
\begin{itemize}
\item para o vetor v = [4, 7, 2, 4, 7, -2, 5, 8, 1, 17]
\end{itemize}
um maior segmento crescente tem comprimento 3.
\begin{itemize}
\item para o vetor v = [10, 10, 5, 3, 2]
\end{itemize}
um maior segmento crescente tem comprimento 1.
\begin{itemize}
\item para o vetor v = [2, 7, 5, 6, 8, 13, 9, 11, 2, 5, 7, 4, 13]
\end{itemize}
um segmento crescente de comprimento máximo tem tamanho 4.

\subsubsection{Questão 4 (3.5 pontos)}
\label{sec:org558a15c}

Dizemos que um número inteiro n é 3-alternante se, quando n é escrito
 na base 3, alterna números pares e ímpares.
Exemplo:
\begin{itemize}
\item 151 é 3-alternante, pois 151 escrito na base 3 é 12121 que alterna pares e ímpares.
\item 145 é 3-alternante, pois escrito na base 3 é 12101, que alterna pares e ímpares.
\item 48 é 3-alternante, pois escrito na base 3 é 1210.
\item 37 não é 3-alternante, pois escrito na base 3 é 1101.
\item 2 é 3-alternante, pois se escreve 2 na base 3.
\end{itemize}
Faça uma função que recebe um inteiro n >= 0 e verifica se n é 3-alternante.
\end{document}

% Created 2020-06-28 dom 22:11
% Intended LaTeX compiler: pdflatex
\documentclass[a4paper]{article}
\usepackage[utf8]{inputenc}
\usepackage[T1]{fontenc}
\usepackage{graphicx}
\usepackage{grffile}
\usepackage{longtable}
\usepackage{wrapfig}
\usepackage{rotating}
\usepackage[normalem]{ulem}
\usepackage{amsmath}
\usepackage{textcomp}
\usepackage{amssymb}
\usepackage{capt-of}
\usepackage{hyperref}
\usepackage[margin=2cm]{geometry}
\usepackage{amsmath,amsfonts,amssymb,amsthm}
\usepackage[utf8]{inputenc}
\usepackage{booktabs}
\usepackage{array}
\usepackage{colortbl}
\usepackage{listings}
\usepackage{graphicx}
\usepackage[english, brazilian]{babel}
\usepackage[scale=2]{ccicons}
\usepackage{hyperref}
\usepackage{relsize}
\usepackage{amsmath}
\usepackage{bm}
\usepackage{amsfonts}
\usepackage{wasysym}
\usepackage{float}
\usepackage{ragged2e}
\usepackage{textcomp}
\usepackage{pgfplots}
\usepackage{todonotes}
\usepgfplotslibrary{dateplot}
\lstdefinelanguage{ein-julia}%
{morekeywords={abstract,struct,break,case,catch,const,continue,do,else,elseif,%
end,export,false,for,function,immutable,mutable,using,import,importall,if,in,%
macro,module,quote,return,switch,true,try,catch,type,typealias,%
while,<:,+,-,::,/},%
sensitive=true,%
alsoother={$},%
morecomment=[l]\#,%
morecomment=[n]{\#=}{=\#},%
morestring=[s]{"}{"},%
morestring=[m]{'}{'},%
}[keywords,comments,strings]%
\lstset{ %
backgroundcolor={},
basicstyle=\ttfamily\scriptsize,
breakatwhitespace=true,
breaklines=true,
captionpos=n,
extendedchars=true,
frame=n,
language=R,
rulecolor=\color{black},
showspaces=false,
showstringspaces=false,
showtabs=false,
stepnumber=2,
stringstyle=\color{gray},
tabsize=2,
}
\renewcommand*{\UrlFont}{\ttfamily\smaller\relax}
\author{Alfredo Goldman}
\date{\today}
\title{Alfredo's MAC0110 Journal}
\hypersetup{
 pdfauthor={Alfredo Goldman},
 pdftitle={Alfredo's MAC0110 Journal},
 pdfkeywords={},
 pdfsubject={},
 pdfcreator={Emacs 26.3 (Org mode 9.1.9)}, 
 pdflang={Brazilian}}
\begin{document}

\maketitle



\subsection{Aula 25 - C para quem programa em Julia}
\label{sec:org901a189}

Como vocês devem lembrar, na primeira aula deixei claro que a linguagem de programação
era apenas o meio de se traduzir algum conceito algorítmico para o computador, de forma
a permitir a sua execução.

Como infelizmente, Julia ainda não é uma linguagem bastante conhecida, várias outras
linguagens mais tradicionais farão compania para vocês no curso. Conhecer várias linguagens
não é ruim, mas por outro lado Julia não ser conhecida é :)

Nessa aula, a ideia é apresentar a linguagem C para quem já programa em Julia. Há algumas
diferenças básicas

\subsubsection{Compilado versus Interpretada}
\label{sec:orgd0dccb9}
 Enquanto Julia é uma linguage interpretada, C é uma linguagem compilada, isso é, a partir
de um código fonte, ao se passar pelo compilador, é gerado um código objeto, que se correto,
pode ser executado na arquitetura para qual foi compilado.

Vejamos um primeiro código em C.
\begin{verbatim}
#include <stdio.h>
int main() {
  printf ("Hello World!\n");
}
\end{verbatim}

Para compilar o código acima, usamos o comando

\begin{verbatim}
gcc um.c
\end{verbatim}
Que vai gerar um arquivo executável a.out, que se chamado imprime
a mensagem. O ponto de entrada inicial de um programa em C é único
e é a função main().

Mas, dá para ver mais umas coisas no código acima que são diferenças
em relação à Julia. Mesmo para coisas básicas como impressão é necessário
incluir bibliotecas, no caso a stdio.h que possui a função printf().

Os blocos são definidos com chaves (abre e fecha) e o ponto e vírgula
delimita os comandos.

Além disso, já dá para ver que C é fortemente tipado, isso é, é necessário
dizer o tipo de tudo ou seja, a função main() acima, não devolve nada.

\subsubsection{Linguagem Tipada}
\label{sec:orgd15b077}

Para cada variável em C, é necessário definir o seu tipo, vamos a mais
um exemplo:

\begin{verbatim}
#include <stdio.h>
 void main() {
   int a = 1;
   int b = 2;
   int c;
   c = a + b
   printf ("O valor de c é %d\n", c);
 }
\end{verbatim}

Acima é possível ver que podemos dar o tipo e definir a variável na mesma linha, ou
declarar e depois usar. Não é possível usar uma variável sem declarar explicitamente.
Isso tem vantagens claras, pois possíveis erros podem ser encontrados já em tempo de
compilação, antes da execução.

O comando de impressão também segue uma sintaxe diferente, recebendo primeiro uma
string, e depois uma lista de parâmetros. Nessa string, para saber como imprimir cada um
dos parâmetros e usar \%, no caso \%d para inteiros, \%g para ponto flutuante e \%s para string.
O barra n  no final é um indicativo para pular linha.

A declaração de funções é semelhante, só que para cada variável passada como parâmetro é
necessário passar o seu tipo. Os tipos mais comuns em C são, int, char, float e double.
Não há o tipo boolean em C, o que se faz é usar comparações, ou tipos inteiros, basicamente
0 equivale a falso e outros valores a verdadeiro.

\begin{verbatim}
#include <stdio.h>

int soma(int a, int b){
  return a + b;
}
void main() {
  int a = 1;
  int b = 2;
  printf ("O valor  é %d\n", soma(a, b));
  if (soma(a, b) == 3)
    printf(" Verdade = %d\n", soma(a, b) == 3);  // Bloco sem chaves
}
\end{verbatim}

No código acima podemos ver que se o bloco tem apenas uma instrução,
não precisa usar as chaves.

Assim como em Julia, a recursão também funciona bem em C.

\subsubsection{Comandos diferentes}
\label{sec:org17afc7e}
Já o for em C é composto por três parâmetros, todos opcionais,
a inicialização, a condição e o passo.

\begin{verbatim}
#include <stdio.h>
void main(){
  for (int p = 1; p <= 512; p *= 2) {
    printf("%d\n", p);
  }
}   
\end{verbatim}


A sintaxe do if é um pouco diferente, principalmente no que se refere ao
uso de elses. Vejamos um exemplo e aproveitemos para usar o comando de 
entrada de dados pelo teclado, o scanf

Comando switch

\begin{verbatim}
#include <stdio.h>
void main(){
  int n;

  printf("Entre com um número: ");
  scanf("%d", &n);
    if (n < 0)
      printf("Número negativo\n");
      else if (n > 0)
        printf("Número positivo\n");
        else
          printf("zero\n");
}    
\end{verbatim}

Observem que quanto mais elses, mais iríamos para a direita, logo isso
se escreve de uma forma alternativa:

\begin{verbatim}
#include <stdio.h>
void main(){
  int n;

  printf("Entre com um número: ");
  scanf("%d", &n);
    if (n < 0)
      printf("Número negativo\n");
    else if (n > 0)
      printf("Número positivo\n");
    else
      printf("zero\n");
}    
\end{verbatim}


Mas, o principal acima é o operador \&, que obtém o endereço de
uma variável, ou seja a sua posição na memória, podemos ver o efeito
disso na seção abaixo.

Mas, antes um exemplo do uso de switch.

\begin{verbatim}
#include <stdio.h>
void main(){
    int n;

    printf("Qual a sua carta (1-13)? ");
    scanf("%d", &n);
    switch (n) \{
      case  1: printf("Ace\n"); break;
      case 11: printf("Jack\n"); break;
      case 12: printf("Queen\n"); break;
      case 13: printf("King\n"); break;
      default: printf("%d\n", n); 
    }
}
\end{verbatim}
\subsubsection{Vetores e ponteiros}
\label{sec:org41f6957}

Vejamos o exemplo abaixo, o primeiro a lidar com ponteiros de forma
mais explícita.

\begin{verbatim}
#include <stdio.h>
void naoModifica(int a){
  a = 2;
}
void Modifica(int *a){
  *a = 2;
}  

void main(){
  int n = 3;

  printf("A variável n vale %d\n", n);
  naoModifica(n);
  printf("A variável n vale %d\n", n);
  Modifica(&n);
  printf("A variável n vale \%d\n", n);   
}    
\end{verbatim}

O mais próximo que vimos de ponteiros em Julia foi o
conceito de vetores, onde um vetor também é um ponteiro,
mas que também guarda o seu tamanho.

Vamos a um exemplo de vetores em C.

\begin{verbatim}
#include <stdio.h>
#include <stdlib.h>

void imprimeVetor(int *v, int tam){
  for (int i = 0; i < tam; i++)
    printf("v[%d] = %d  ",i, v[i]);
  printf("\n");
}

void inicializaVetor(int *v, int tam){
  for (int i = 0; i < tam; i++)
    v[i] = rand() % 100;
}

void vezes2Vetor(int *v, int tam){
  for (int i = 0; i < tam; i++)
    v[i] = 2 * v[i];
}



void main()
{
  int vetor[10];
  inicializaVetor(vetor, 10);
  imprimeVetor(vetor, 10);
  vezes2Vetor(vetor, 10);
  imprimeVetor(vetor, 10);  
}
\end{verbatim}

Mas, com grandes poderes vem grandes responsabilidades, veja o programa abaixo com
uma pequena modificação e um erro inserido.

\begin{verbatim}
#include <stdio.h>
#include <stdlib.h>

void imprimeVetor(int *v, int tam){
  for (int i = 0; i < tam; i++)
    printf("v[%d] = %d  ",i, v[i]);
  printf("\n");
}

void inicializaVetor(int *v, int tam){
  for (int i = 0; i < tam; i++)
    v[i] = rand() % 100;
}

void vezes2Vetor(int *v, int tam){
  for (int i = 0; i < tam; i++)
    v[i] = 2 * v[i];
}



void main()
{
  int vetor[10];
  int *ptr;

  ptr = vetor;
  inicializaVetor(ptr, 10);
  imprimeVetor(ptr, 10);
  vezes2Vetor(ptr, 10);
  ptr++;
  imprimeVetor(ptr, 10);

}
\end{verbatim}
ou
\begin{verbatim}
#include <stdio.h>
#include <stdlib.h>

void imprimeVetor(int *v, int tam){
  for (int i = 0; i < tam; i++)
    printf("v[%d] = %d  ",i, v[i]);
  printf("\n");
}

void inicializaVetor(int *v, int tam){
  for (int i = 0; i < tam; i++)
    v[i] = rand() % 100;
}

void vezes2Vetor(int *v, int tam){
  for (int i = 0; i < tam; i++)
    v[i] = 2 * v[i];
}



void main()
{
  int *ptr;


  ptr = malloc(10 * sizeof(int));
  inicializaVetor(ptr, 10);
  imprimeVetor(ptr, 11);
  vezes2Vetor(ptr, 11);
  imprimeVetor(ptr, 11);
}
\end{verbatim}
\end{document}

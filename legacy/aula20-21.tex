% Created 2020-06-03 qua 08:05
% Intended LaTeX compiler: pdflatex
\documentclass[a4paper]{article}

\usepackage{booktabs}
\usepackage[margin=2cm]{geometry}
\usepackage{amsmath,amsfonts,amssymb,amsthm}
\usepackage{sourcecodepro}
\usepackage[utf8]{inputenc}
\usepackage{booktabs}
\usepackage{array}
\usepackage{colortbl}
\usepackage{listings}
\usepackage{algpseudocode}
\usepackage{algorithm}
\usepackage{graphicx}
\usepackage[english, ]{babel}
\usepackage[scale=2]{ccicons}
\usepackage{hyperref}
\usepackage{relsize}
\usepackage{amsmath}
\usepackage{bm}
\usepackage{amsfonts}
\usepackage{wasysym}
\usepackage{float}
\usepackage{ragged2e}
\usepackage{textcomp}
\usepackage{pgfplots}
\usepackage{todonotes}
\usepgfplotslibrary{dateplot}
\lstdefinelanguage{ein-julia}%
{morekeywords={abstract,struct,break,case,catch,const,continue,do,else,elseif,%
end,export,false,for,function,immutable,mutable,using,import,importall,if,in,%
macro,module,quote,return,switch,true,try,catch,type,typealias,%
while,<:,+,-,::,/},%
sensitive=true,%
alsoother={$},%
morecomment=[l]\#,%
morecomment=[n]{\#=}{=\#},%
morestring=[s]{"}{"},%
morestring=[m]{'}{'},%
}[keywords,comments,strings]%
\lstset{ %
backgroundcolor={},
basicstyle=\ttfamily\scriptsize,
breakatwhitespace=true,
breaklines=true,
captionpos=n,
extendedchars=true,
frame=n,
language=R,
rulecolor=\color{black},
showspaces=false,
showstringspaces=false,
showtabs=false,
stepnumber=2,
stringstyle=\color{gray},
tabsize=2,
}
\renewcommand*{\UrlFont}{\ttfamily\smaller\relax}
\author{Alfredo Goldman}
\date{\today}
\title{Alfredo's MAC0110 Journal}
\hypersetup{
 pdfauthor={Alfredo Goldman},
 pdftitle={Alfredo's MAC0110 Journal},
 pdfkeywords={},
 pdfsubject={},
 pdfcreator={Emacs 26.3 (Org mode 9.3.6)},
 pdflang={Bt-Br}}
\begin{document}

\maketitle

\subsection{Aula 20 - Ainda ordenação}
\label{sec:org693b0b7}

Nessa aula vamos continuar vendo a ordenação, mas antes disso,
vamos finalmente ver o algoritmo de busca binária para
encontrar um elemento em um vetor ordenado.

\subsubsection{Busca binária}
\label{sec:orgb9f7cad}
Ao invés de percorrer o vetor, desde o início, procurando o
elemento como em:
\lstset{language=ein-julia,label= ,caption= ,captionpos=b,numbers=none}
\begin{lstlisting}
function buscaLinear(x, v)
  for i in 1:length(v)
    if v[i] == x
      return i
    end
  end
  return 0
end
\end{lstlisting}

Podemos a cada procura, para ver se o elemento está
no vetor, eliminar metado do vetor. Se o elemento for
menor que o meio, olhamos do começo ao meio. se for maior
que o meio, olhamos do meio ao fim. Se for igual, achou.

\lstset{language=ein-julia,label= ,caption= ,captionpos=b,numbers=none}
\begin{lstlisting}
function buscaBinaria(x, v)
   inicio = 1
   fim = length(v)
   while in <= fim
      meio = div(inicio + fim, 2)
      if v[meio] == x
        return meio
      elseif x < v[meio]
        fim = meio - 1
      else
        inicio = meio + 1
      end
   end
   return 0
end
\end{lstlisting}

 Há também a versão recursiva, para isso temos que ter
o inĩcio e o fim como parâmetros.

\lstset{language=ein-julia,label= ,caption= ,captionpos=b,numbers=none}
\begin{lstlisting}
function buscaBinariaRec(inicio, fim, x, v)
   if inicio > fim
     return 0
   end
   meio = div(inicio + fim, 2)
   if v[meio] == x
     return meio
   elseif x < v[meio]
     buscaBinariaRec(inicio, meio - 1, x, v)
   else
     buscaBinariaRec(meio + 1, fim, x, v)
   end
end
\end{lstlisting}

\subsubsection{Métodos de busca mais elaborados}
\label{sec:org2be7e9e}

Uma forma mais eficiente de se ordenar um vetor é usando a divisão e conquista,
isso é, dado um vetor, quebramos em duas partes, ordenamos as partes e depois
fazemos o merge, no caso como podemos ter repetição, vamos usar a versão que
permite duplicação.

\lstset{language=ein-julia,label= ,caption= ,captionpos=b,numbers=none}
\begin{lstlisting}
function merge(u, v)
  pu = 1  # ponteiro em u
  pv = 1  # ponteiro em v
  resp = []
  while pu <= length(u) && pv <= length(v)
    if u[pu] < v[pv]
      push!(resp, u[pu])
      pu = pu + 1
    elseif v[pv] < u[pu]
      push!(resp, v[pv])
      pv = pv + 1
    else
      push!(resp, v[pv])
      pu = pu + 1
    end
  end
  while pu <= length(u)
    push!(resp, u[pu])
    pu = pu + 1
  end
  while pv <= length(v)
    push!(resp, v[pv])
    pv = pv + 1
    end
  return resp
end
\end{lstlisting}

Dado o merge, a ideia é:
\begin{itemize}
\item Divida o vetor no meio
\item Ordene cada metade separadamente
\item Devolva o merge
\end{itemize}

Para isso vamos ver mais uma possibilidade de Julia,
dado um vetor v, v[1:meio] cria um vetor até o meio e
v[meio + 1:length(v)] cria um vetor do meio + 1 ao final.

Com isso fica fácil fazer o mergeSort.

\lstset{language=ein-julia,label= ,caption= ,captionpos=b,numbers=none}
\begin{lstlisting}
function mergeSort(v)
  if length(v) <= 1
    return v
  end
  meio = div(length(v), 2)
  v1 = v[1:meio]
  v2 = v[meio + 1:length(v)]
  v1ord = mergeSort(v1)
  v2ord = mergeSort(v2)
  return merge(v1ord, v2ord)
end
\end{lstlisting}

 Vamos ao último algoritmo, que também fica melhor de forma
recursiva, dado um vetor, o primeiro passo é escolher um elemento
 de forma a dividir o vetor em duas partes, quem for menor ou
igual a esse elemento, e quem for maior. Isso deve ser feito
de forma recursiva.

\lstset{language=ein-julia,label= ,caption= ,captionpos=b,numbers=none}
\begin{lstlisting}
function quick!(i, j, v)
    if j > i
        pivo = v[div(i+j, 2)]
        left = i
        right = j
        while left <= right
            while v[left] < pivo
                left += 1
            end
            while v[right] > pivo
                right -= 1
            end
            if left <= right
                v[left], v[right] = v[right], v[left]
                left += 1
                right -= 1
            end
        end
        quick!(i, right, v)
        quick!(left, j, v)
    end
    return v
end
\end{lstlisting}
\newpage
\subsection{Aula 21 - Indo além de vetores}
\label{sec:org61260c0}
Nessa aula o objetivo é ir além de vetores, primeiro entendendo que em
Júlia o que se pode fazer com vetores é bem flexível, em seguida vamos
ver que vetores podem ter várias dimensões. Veremos ao final alguns algoritmos
com matrizes.

\subsubsection{Indo além de vetores}
\label{sec:orgedfa311}
Conforme já vimos antes, os vetores ocupam várias posições de
memória, uma forma de se inicializar um vetor é através da função
zeros. Que cria um vetor com valores nulos.

\lstset{language=ein-julia,label= ,caption= ,captionpos=b,numbers=none}
\begin{lstlisting}
zeros(1)
zeros(100)
zeros(Int8, 3)
\end{lstlisting}

A novidade é que agora o zeros pode criar também vetores de mais
de uma dimensão, ou matrizes.
\lstset{language=ein-julia,label= ,caption= ,captionpos=b,numbers=none}
\begin{lstlisting}
zeros(3,3)
zeros(Int, 2, 2)
\end{lstlisting}

Duas outras funçÕes interessantes são a ones() com comportamento
similar a zeros(), e a rand() que cria vetores ou matrizes com
números aleatórios.

O comando fill() serve para fazer isso com um valor específico.
\lstset{language=ein-julia,label= ,caption= ,captionpos=b,numbers=none}
\begin{lstlisting}
ones(3, 3)
rand(4, 4)
rand(1: 10, 5, 5)
fill(42, 3, 3, 3)
\end{lstlisting}

Também é possível criar matrizes diretamente.

\lstset{language=ein-julia,label= ,caption= ,captionpos=b,numbers=none}
\begin{lstlisting}
m = [1 2 3; 4 5 6; 7 8 9]
\end{lstlisting}

O comando length() funciona para matrizes, mas devolve o seu tamanho
total, mas temos também os comandos ndims() e size().


A linguagem Julia ainda oferece vários açucares sintáticos, vejamos o
exemplo abaixo.
\lstset{language=ein-julia,label= ,caption= ,captionpos=b,numbers=none}
\begin{lstlisting}
a = [1 2; 3 4]
b = [1 0; 0 1]
a * b
a .* b
\end{lstlisting}

Mas, lembrando que o objetivo da disciplina é ensinar algoritmos, vamos
ver alguns.

Uma função que imprime uma matriz quadrada m.

\lstset{language=ein-julia,label= ,caption= ,captionpos=b,numbers=none}
\begin{lstlisting}
function imprimeMatriz(m)
    a = size(m)
    soma = 0
    for i in 1:a[1]
       for j in 1:a[2]
          print(" m[", i, ", ", j, "] = ", m[i,j])
        end
        println()
    end
end
\end{lstlisting}


Faça uma função que recebe uma matriz quadrada e devolve a soma dos seus
elementos.

\lstset{language=ein-julia,label= ,caption= ,captionpos=b,numbers=none}
\begin{lstlisting}
function somaMatriz(m)
    a = size(m)
    soma = 0
    for i in 1:a[1]
       for j in 1:a[2]
          soma = soma + m[i, j]
        end
     end
     return soma
  end
\end{lstlisting}

Uma matriz quadrada é um quadrado mágico se a soma dos elementos de
cada linha, de cada coluna e das diagonais é sempre igual. Faça uma
função que dada uma matriz quadrada m, verifica se ela é um
quadrado mágico.

\lstset{language=ein-julia,label= ,caption= ,captionpos=b,numbers=none}
\begin{lstlisting}
function QuadradoMagico(m)
    if length(m) == 1
      return true
    end
    a = size(m)
    if length(a) != 2 || a[1] != a[2]
      return false
    end
    somaP = 0
    for i in 1:a[1]
      somaP = somaP + m[i, i]
    end
    soma = 0
    for i in 1:a[1]
       soma = soma + m[i, a[1] - i + 1]
    end
    if somaP != soma
       return false
    end
    for i in 1:a[1]
       somaL = 0
       somaC = 0
       for j in 1:a[2]
          somaL = somaL + m[i, j]
          somaC = somaC + m[j, i]
       end
       if somaL != somaP || somaC != somaP
          return false
       end
    end
    return true
end
\end{lstlisting}

 Dadas duas matrizes m e n, faça uma função que devolva o
produto delas (sem usar o * para matrizes).

\lstset{language=ein-julia,label= ,caption= ,captionpos=b,numbers=none}
\begin{lstlisting}
function multiplica(a, b)
   dima = size(a)
   dimb = size(b)
   if dima[2] != dimb[1]
     return -1
   end
   c = zeros(dima[1], dimb[2])
   for i in 1:dima[1]
      for j in 1:dimb[2]
         for k in 1:dima[2]
            c[i, j] = c[i, j] + a[i, k] * b[k, j]
         end
       end
   end
   return c
 end

\end{lstlisting}

Uma matriz quadrada de tamanho n é um quadrado latino se em cada linha e coluna
aparecem todos os valores de 1 a n. Faça uma função que dada uma matriz
quadrada verifica se ela é um quadrado latino.
\end{document}
